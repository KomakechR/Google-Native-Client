<<<<<<< HEAD
\documentclass {article}
\usepackage{float}


\usepackage[a4paper, total={6in, 8in}]{geometry}



\begin{document}

\title{ NATIVE CLIENT}
\author{KOMAKECH RONALD 15/u/6690/EVE 215011576}

\maketitle

\section{Introduction}
Native Client is a browser plug-in that lets websites execute compiled, native C and C++ code \cite{r1} . The browser plugin which is common with Google’s chrome browser \cite{r6} was developed using Native Client (NaCl) an open-source technology which runs native compiled code in the browser at near-native speeds. The open source technology gained the trust and popular vote of programmers for its much needed attributes of maintaining the portability and safety \cite{r7} that users expect from web applications. Native Client comes in two flavors: traditional (NaCl) and portable (PNaCl). Traditional, which must be distributed through the Chrome Web Store lets you target a specific hardware platform while portable can run on the open web. 
\par
Having introduced what native client is, one may still wonder how it really works. Well, a bitcode file is downloaded to a client computer and converted to hardware-specific code before any execution \cite{r2}. Over time, the technology has been made more user friendly hence the increase in the number of users as well as native client products namely; and a number of products have 

\par
Among the notable benefits of Native Client are;\newline
It gives web applications some advantages of desktop software. Specifically, it provides the means to fully harness the client’s computational resources for applications such as expanding web programming beyond JavaScript, enabling programmers to enhance web applications using their preferred language. A Native Client web application basically consists of JavaScript, HTML, CSS, and a NaCl module written in a language supported by the SDK.\newline
It also gives C and C++ (and other languages targeting it) the same level of portability and safety as JavaScript.\newline
It also offers portability as the applications written in Native Client can be run on multiple operating systems (Windows, Linux, Mac, and Chrome OS) and CPU architectures (x86 and ARM).\newline
It also offers an easy migration path to the web: Leveraging years of work in existing desktop applications. Native Client makes the transition from the desktop to a web application significantly easier because it supports C and C++.\newline
It offers security by protecting the user’s system from malicious applications through its double sandbox model. This model offers the safety of traditional web applications without sacrificing performance and without requiring users to install a plug-in.


\newpage
\nocite{*}
\bibliographystyle{IEEEtran}
\bibliography{bibliography}


=======
\documentclass {article}

\begin{document}

\title{ NATIVE CLIENT}
\author{KOMAKECH RONALD 15/u/6690/EVE 215011576}

\maketitle

\section{Introduction}
Native Client is a browser plug-in that lets websites execute compiled, native C and C++ code. The browser plugin which is common with Google’s chrome browser was developed using Native Client (NaCl) an open-source technology which runs native compiled code in the browser at near-native speeds. The open source technology gained the trust and popular vote of programmers for its much needed attributes of maintaining the portability and safety that users expect from web applications. Native Client comes in two flavors: traditional (NaCl) and portable (PNaCl). Traditional, which must be distributed through the Chrome Web Store lets you target a specific hardware platform while portable can run on the open web. 
\par
Having introduced what native client is, one may still wonder how it really works. Well, a bitcode file is downloaded to a client computer and converted to hardware-specific code before any execution. Over time, the technology has been made more user friendly hence the increase in the number of users as well as native client products namely; and a number of products have 

\par
Among the notable benefits of Native Client are;
It gives web applications some advantages of desktop software. Specifically, it provides the means to fully harness the client’s computational resources for applications such as expanding web programming beyond JavaScript, enabling programmers to enhance web applications using their preferred language. A Native Client web application basically consists of JavaScript, HTML, CSS, and a NaCl module written in a language supported by the SDK.
It also gives C and C++ (and other languages targeting it) the same level of portability and safety as JavaScript.



\nocite{*}
\bibliographystyle{ieee}
\bibliography{references}

>>>>>>> 9855fd4148d0b95e2c9b5749b6fc36de950183f0
\end{document}